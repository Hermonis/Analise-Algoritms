\documentclass[a4paper,14pt]{extreport}

\usepackage{cmap} % Улучшенный поиск русских слов в полученном pdf-файле
\usepackage[T2A]{fontenc} % Поддержка русских букв
\usepackage[utf8]{inputenc} % Кодировка utf8
\usepackage[english,russian]{babel} % Языки: русский, английский
%\usepackage{pscyr} % Нормальные шрифты

\usepackage{amsmath}

\usepackage{geometry}
\geometry{left=30mm}
\geometry{right=15mm}
\geometry{top=20mm}
\geometry{bottom=20mm}

\usepackage{titlesec}
\titleformat{\section}
{\normalsize\bfseries}
{\thesection}
{1em}{}
\titlespacing*{\chapter}{0pt}{-30pt}{8pt}
\titlespacing*{\section}{\parindent}{*4}{*4}
\titlespacing*{\subsection}{\parindent}{*4}{*4}

\usepackage{setspace}
\onehalfspacing % Полуторный интервал

\frenchspacing
\usepackage{indentfirst} % Красная строка

\usepackage{titlesec}
\titleformat{\chapter}{\LARGE\bfseries}{\thechapter}{20pt}{\LARGE\bfseries}
\titleformat{\section}{\Large\bfseries}{\thesection}{20pt}{\Large\bfseries}

\usepackage{listings}
\usepackage{xcolor}


\lstdefinestyle{cpp}{
	language=C++,
	backgroundcolor=\color{white},
	basicstyle=\footnotesize\ttfamily,
	keywordstyle=\color{blue},
	stringstyle=\color{red},
	commentstyle=\color{gray},
	directivestyle=\color{orange}
	numbers=left,
	numberstyle=\tiny,
	stepnumber=1,
	numbersep=5pt,
	frame=single,
	tabsize=8,
	captionpos=b,
	breaklines=true,
	breakatwhitespace=true,
	escapeinside={\#*}{*)},
	morecomment=[l][\color{magenta}]{\#},
	columns=fullflexible
}

\usepackage{pgfplots}
\usetikzlibrary{datavisualization}
\usetikzlibrary{datavisualization.formats.functions}

\usepackage{graphicx}
\newcommand{\img}[3] {
	\begin{figure}[h]
		\center{\includegraphics[height=#1]{inc/img/#2}}
		\caption{#3}
		\label{img:#2}
	\end{figure}
}
\newcommand{\boximg}[3] {
	\begin{figure}[h]
		\center{\fbox{\includegraphics[height=#1]{inc/img/#2}}}
		\caption{#3}
		\label{img:#2}
	\end{figure}
}
\newcommand{\imgext}[4] {
	\begin{figure}[h!]
		\center{\includegraphics[#1]{inc/img/#2.#3}}
		\caption{#4}
		\label{img:#2}
	\end{figure}
}

\usepackage[justification=centering]{caption} % Настройка подписей float объектов

\usepackage[unicode,pdftex]{hyperref} % Ссылки в pdf
\hypersetup{hidelinks}

\newcommand{\code}[1]{\texttt{#1}}

\usepackage{icomma} % Интеллектуальные запятые для десятичных чисел

\usepackage{csvsimple}

\usepackage{svg}

\usepackage{algorithm} % Псевдокод
\usepackage[noend]{algpseudocode}
\floatname{algorithm}{Листинг}

\usepackage{booktabs}


\begin{document}

\begin{titlepage}
	\centering
	
	{\footnotesize\itshape Министерство науки и высшего образования
		Российской Федерации Федеральное государственное бюджетное
		образовательное учреждение высшего образования «Московский
		государственный технический университет имени Н.~Э.~Баумана
		(национальный исследовательский университет)» (МГТУ им. Н.~Э.~Баумана)
		\\}
	
	\vspace{60mm}
	
	\textbf{ОТЧЕТ}\\
	По лабораторной работе №6\\
	По курсу: «Анализ алгоритмов»\\
	Тема: «Муравьиный алгоритм»\\
	
	\vspace{60mm}
	
	\hspace{70mm} Студент:\hfill Козаченко~А.~А.\\
	\hspace{70mm} Группа: \hfill ИУ7-54Б\\
	%\vspace{5mm}
	\hspace{70mm} Преподаватели:\hfill Волкова~Л.~Л.,\\
	\hfill Строганов~Ю.~В.\\
	
	\vfill
	Москва, 2020
\end{titlepage}


\tableofcontents

\chapter*{Введение}
\addcontentsline{toc}{chapter}{Введение}

Муравьиный алгоритм — алгоритм для нахождения приближённых решений задач оптимизации на графах, таких, как задача коммивояжера, транспортная задача и аналогичных задач поиска маршрутов на графах.
Муравья нельзя назвать сообразительным.
Отдельный муравей не в состоянии принять ни малейшего решения.
Дело в том, что он устроен крайне примитивно: все его действия сводятся к элементарным реакциям на окружающую обстановку и своих собратьев.
Муравей не способен анализировать, делать выводы и искать решения.

Эти факты, однако, никак не согласуются с успешностью муравьев как вида.
Они существуют на планете более 100 миллионов лет, строят огромные жилища, обеспечивают их всем необходимым и даже ведут настоящие войны.
В сравнении с полной беспомощностью отдельных особей, достижения муравьев кажутся немыслимыми.

Добиться таких успехов муравьи способны благодаря своей социальности.
Они живут только в коллективах – колониях. Все муравьи колонии формируют так называемый роевой интеллект.
Особи, составляющие колонию, не должны быть умными: они должны лишь взаимодействовать по определенным – крайне простым – правилам, и тогда колония целиком будет эффективна.

В колонии нет доминирующих особей, нет начальников и подчиненных, нет лидеров, которые раздают указания и координируют действия.
Колония является полностью самоорганизующейся.
Каждый из муравьев обладает информацией только о локальной обстановке, не один из них не имеет представления обо всей ситуации в целом – только о том, что узнал сам или от своих сородичей, явно или неявно.
На неявных взаимодействиях муравьев, называемых стигмергией, основаны механизмы поиска кратчайшего пути от муравейника до источника пищи.

Каждый раз проходя от муравейника до пищи и обратно, муравьи оставляют за собой дорожку феромонов.
Другие муравьи, почувствовав такие следы на земле, будут инстинктивно устремляться к нему.
Поскольку эти муравьи тоже оставляют за собой дорожки феромонов, то чем больше муравьев проходит по определенному пути, тем более привлекательным он становится для их сородичей.
При этом, чем короче путь до источника пищи, тем меньше времени требуется муравьям на него – а следовательно, тем быстрее оставленные на нем следы становятся заметными.

В 1992 году в своей диссертации Марко Дориго (Marco Dorigo) предложил заимствовать описанный природный механизм для решения задач оптимизации \cite{Dorigo}.
Имитируя поведение колонии муравьев в природе, муравьиные алгоритмы используют многоагентные системы, агенты которых функционируют по крайне простым правилам.
Они крайне эффективны при решении сложных комбинаторных задач – таких, например, как задача коммивояжера \cite{TSPr}, первая из решенных с использованием данного типа алгоритмов \cite{Mueller}.

\section*{Задачи работы}

Цель лабораторной работы: изучить муравьиный алгоритм на материале решения задачи Коммивояжера.

В рамках выполнения работы необходимо решить следующие задачи:
\begin{itemize}
	\item дать постановку задачи;
	\item описать методы полного перебора и эвристический, основанный на муравьином алгоритме;
	\item реализовать данные методы;
	\item выбрать и подготовить классы данных;
	\item провести параметризацию метода, основанного на муравьином алгоритме;
	\item интерпретировать результаты и сравнить их с результатами метода полного перебора.
\end{itemize}


\chapter{Аналитическая часть}

\section{Описание алгоритмов}

\subsubsection{Алгоритм полного перебора}

Алгоритм полного перебора для решения задачи коммивояжера предполагает рассмотрение всех возможных путей в графе и выбор наименьшего из них.

Такой подход гарантирует точное решение задачи, однако, так как задача относится к числу трансвычислительных, то уже при небольшом числе городов решение за приемлемое время невозможно.

\subsubsection{Муравьиный алгоритм}

Идея алгоритма основана на принципе работы колонии муравьев \cite{Bonabeau}. Колония муравьев рассматривается как многоагентная система, в которой каждый агент (муравей) функционирует автономно по очень простым правилам. В противовес почти примитивному поведению агентов, поведение всей системы получается разумным.

Каждый муравей определяет для себя маршрут, который необходимо пройти на основе феромона, который он ощущает, во время прохождения, каждый муравей оставляет феромон на своем пути, чтобы остальные муравьи могли по нему ориентироваться. В результате при прохождении каждым муравьем различного маршрута наибольшее число феромона остается на оптимальном пути. 

Самоорганизация колонии является результатом взаимодействия следующих компонентов:
\begin{itemize}
	\item случайность — муравьи имеют случайную природу движения;
	\item многократность — колония допускает число муравьев, достигающее от нескольких десятков до миллионов особей;
	\item положительная обратная связь — во время движения муравей откладывает феромон, позволяющий другим особям определить для себя оптимальный маршрут;
	\item отрицательная обратная связь — по истечении определенного времени феромон испаряется;
	\item целевая функция.
\end{itemize}

Пусть муравей обладает следующими характеристиками:
\begin{itemize}
	\item зрение — определяет длину ребра;
	\item обоняние — чувствует феромон;
	\item память — запоминает маршрут, который прошел.
\end{itemize}

Введем целевую функцию $\eta_{ij} = 1 / D_{ij}$, где $D_{ij}$ — расстояние из текущего пункта $i$ до заданного пункта $j$.

Посчитаем вероятности перехода в заданную точку по формуле \eqref{possibility}:
\begin{equation}
	\label{possibility}
	P_{kij} = \begin{cases}
		\frac{t_{ij}^a\eta_{ij}^b}{\sum_{q=1}^m t^a_{iq}\eta^b_{iq}}, \textrm{вершина не была посещена ранее муравьем k,} \\
		0, \textrm{иначе}
	\end{cases}
\end{equation}
где $a, b$ -- настраиваемые параметры, $t$ - концентрация феромона, причем $a + b = const$, а при $a = 0$ алгоритм вырождается в жадный \cite{Levitin}.

Когда все муравьи завершили движение происходит обновление феромона по формуле \eqref{pheromone1}:
\begin{equation}
	\label{pheromone1}
	t_{ij}(t+1) = (1-p)t_{ij}(t) + \Delta t_{ij}, \Delta t_{ij} = \sum_{k=1}^N t^k_{ij}
\end{equation}
где
\begin{equation}
	\label{pheromone2}
	\Delta t^k_{ij} = \begin{cases}
		Q/L_{k}, \textrm{ребро посещено k-ым муравьем,} \\
		0, \textrm{иначе}
	\end{cases}
\end{equation}
$L_{k}$ — длина пути k-ого муравья, $Q$ — настраивает концентрацию нанесения/испарения феромона, $N$ — количество муравьев.

\section*{Вывод}
По итогам аналитического раздела были описаны алгоритмы полного перебора и муравьиный алгоритм.


\chapter{Конструкторская часть}

\section{Разработка алгоритмов}

\subsubsection{Алгоритм полного перебора}

В листинге \ref{lst:pseudo-full} приведен псевдокод алгоритма полного перебора для решения задачи коммивояжера.
\begin{algorithm}
	\caption{\label{lst:pseudo-full}Алгоритм полного перебора для решения задачи коммивояжера $ex\_search(G, E)$}
	\begin{algorithmic}
		\State $best\_path\_len \gets -1$
		\State $queue \gets (0, 0)$
		\While{$queue \text{ is not empty} $}
		\State $ u <- queue.pop() $
		\For{$v \text{ in } E(u[u.size()-1])$}
		\If {$v \text{ in } u$}
		\State $continue$
		\EndIf
		\State $ new\_path \gets u $
		\State $ new\_path.add(v) $
		\State $ update\_best\_path $
		\EndFor
		\EndWhile
	\end{algorithmic}
\end{algorithm}

На рис. \ref{img:ant} отображена работа одного муравья.

\imgext{height=90mm}{ant}{pdf}{Схема работы одного муравья}

Описание этапов:
\begin{itemize}
	\item инициализация муравья — установка муравья в стартовый город;
	\item цикл продолжает работу до тех пор, пока все вершины не будут посещены, внутри тела цикла высчитывается вероятность посещения следующего города по формуле (\ref{possibility}). Для определения конкретного города применяется метод рулетки, в котором случайно выбирается значение $ 0, \dots , 1 $ и на основе полученного значения с учетом вероятностей перехода в непосещённые города определяется следующий город.
\end{itemize}

Описанный выше алгоритм применяется $N$ раз, где $N$ — количество вершин.
Каждый муравей помещается в отдельную вершину на карте, после чего ищет маршрут.
Когда последний муравей завершил обход всех вершин, производится поиск оптимального из полученных маршрутов, после чего феромон обновляется и в случае, если полученный результат не удовлетворяет поставленной задаче, алгоритм запускается заново.

На рис.\ref{img:ants} представлена работа алгоритма для всей колонии.

\imgext{height=110mm}{ants}{pdf}{Схема работы колонии}

\section*{Вывод}

В результате работы над конструкторским разделом была разработана схемы алгоритма полного перебора и муравьиного алгоритма.


\chapter{Технологическая часть}

В данном разделе приведены средства реализации и листинг кода.

\section{Требования к ПО}

Программа должна обрабатывать матрицу смежностей методами полного перебора и эвристическим, основанным на муравьином алгоритме, подбирать параметры, на основе которых производятся оптимальные вычисления, выводить заданную матрицу смежностей, результат работы полного перебора и муравьиного алгоритма.

\section{Средства реализации}

В качестве языка программирования для реализации данной лабораторной работы был выбран высокопроизводительный язык C++ \cite{cpp17}, так как он предоставляет широкие возможности для эффективной реализации алгоритмов.

\section{Листинг кода}

В листингах \ref{lst:exhaustive} и \ref{lst:colony} представлены алгоритм полного перебора и класса колонии соответственно.

\begin{lstinputlisting}[
	caption={Алгоритм полного перебора},
	label={lst:exhaustive},
	style={cpp},
	linerange={16-45}
	]{../source/algs/exhaustive_search.cpp}
\end{lstinputlisting}

\begin{lstinputlisting}[
	caption={Класс муравьиной колонии},
	label={lst:colony},
	style={cpp},
	linerange={7-41}
	]{../source/algs/colony.hpp}
\end{lstinputlisting}

\section*{Вывод}

Правильный выбор инструментов разработки позволил эффективно реализовать алгоритмы, настроить модульное тестирование и выполнить исследовательский раздел лабораторной работы.


\chapter{Исследовательская часть}

\section{Технические характеристики}

\begin{itemize}
	\item Операционная система: Ubuntu 19.10 64-bit.
	\item Память: 3,8 GiB.
	\item Процессор: Intel® Core™ i3-6006U CPU @ 2.00GHz
\end{itemize}

\section{Постановка эксперимента}

В муравьином алгоритме вычисления производятся на основе настраиваемых параметров.
Рассмотрим два класса данных и подберем к ним параметры, при которых метод даст точный результат при минимальном количестве итераций.

Будем рассматривать матрицы размерности $10\times10$, так как иначе получение точного результата алгоритмом полного перебора слишком велико.

В качестве первого класса данных выделим матрицу смежностей, в которой все значения незначительно отличаются друг от друга, например, в диапазоне $[0, 10]$.
Вторым классом будут матрицы, где значения могут значительно отличаться, например $[1, 15000]$.

Будем запускать муравьиный алгоритм для всех значений $\alpha, P\in[0, 1]$, с шагом $= 0.1$, пока не будет найдено точное значение для каждого набора.
Если будет превышено допустимое количество итераций работа алгоритма при данных параметров будет завершена.

В результате тестирования будет выведена таблица со значениями $\alpha, \beta, P$, $iters$, $ dist$, где $iters$ — число итераций, за которое алгоритм нашел оптимальный путь, $dist$ — длина найденного пути, а $\alpha, \beta, p$ — настроечные параметры.

Ниже будут представлены результаты работы алгоритма для двух классов данных.

\subsection{Класс данных 1}

Матрица смежности для класса данных 1:
\begin{equation*}
	G = \begin{pmatrix}
		0,      &3      &10     &9      &8      &2      &1      &7      &2      &10     \\
		3,      &0      &8      &10     &3      &9      &6      &8      &4      &2      \\
		10,     &8      &0      &10     &3      &2      &7      &3      &10     &7      \\
		9,      &10     &10     &0      &3      &2      &2      &5      &10     &4      \\
		8,      &3      &3      &3      &0      &3      &6      &8      &5      &1      \\
		2,      &9      &2      &2      &3      &0      &10     &10     &2      &9      \\
		1,      &6      &7      &2      &6      &10     &0      &10     &4      &8      \\
		7,      &8      &3      &5      &8      &10     &10     &0      &6      &2      \\
		2,      &4      &10     &10     &5      &2      &4      &6      &0      &10     \\
		10,     &2      &7      &4      &1      &9      &8      &2      &10     &0
	\end{pmatrix}
\end{equation*}

В таблице ~\ref{T:log111} приведены результаты параметризаций метода решения задачи коммивояжера на основании муравьиного алгортима. Полный перебор определил оптимальную длину пути 22.

\begin{table}
	\caption{Таблица коэффициентов для класса данных №1}
	\begin{minipage}[h!]{0.10\hsize}\centering
		\begin{center}\resizebox{4\textwidth}{!}{%
				\begin{tabular}{c@{\hspace{5mm}}c@{\hspace{5mm}}c@{\hspace{5mm}}c@{\hspace{5mm}}c@{\hspace{5mm}}c}
					\toprule
					a        &b      &p      &iters &длина пути \\
					\midrule
					0       &1      &0      &50    &22\\
					0       &1      &0.1    &50    &22\\
					0       &1      &0.2    &50    &22\\
					0       &1      &0.3    &50    &22\\
					0       &1      &0.4    &50    &22\\
					0       &1      &0.5    &50    &22\\
					0       &1      &0.6    &50    &22\\
					0       &1      &0.7    &50    &22\\
					0       &1      &0.8    &50    &22\\
					0       &1      &0.9    &50    &22\\
					0       &1      &1      &7     &22\\
					\midrule
					0.1     &0.9    &0      &7     &22\\
					0.1     &0.9    &0.1    &24    &22\\
					0.1     &0.9    &0.2    &174   &22\\
					0.1     &0.9    &0.3    &174   &22\\
					0.1     &0.9    &0.4    &24    &23\\
					0.1     &0.9    &0.5    &24    &22\\
					0.1     &0.9    &0.6    &24    &22\\
					0.1     &0.9    &0.7    &174   &22\\
					0.1     &0.9    &0.8    &174   &22\\
					0.1     &0.9    &0.9    &119   &23\\
					0.1     &0.9    &1      &22    &23\\
					\midrule
					0.2     &0.8    &0      &24    &22\\
					0.2     &0.8    &0.1    &24    &23\\
					0.2     &0.8    &0.2    &24    &22\\
					0.2     &0.8    &0.3    &24    &22\\
					0.2     &0.8    &0.4    &174   &22\\
					0.2     &0.8    &0.5    &174   &22\\
					0.2     &0.8    &0.6    &24    &22\\
					0.2     &0.8    &0.7    &24    &22\\
					0.2     &0.8    &0.8    &24    &22\\
					0.2     &0.8    &0.9    &30    &22\\
					0.2     &0.8    &1      &2     &22\\
					\midrule
					0.3     &0.7    &0      &20    &22\\
					0.3     &0.7    &0.1    &20    &22\\
					0.3     &0.7    &0.2    &20    &22\\
					0.3     &0.7    &0.3    &20    &22\\
					0.3     &0.7    &0.4    &20    &22\\
					0.3     &0.7    &0.5    &20    &22\\
					0.3     &0.7    &0.6    &20    &23\\
					0.3     &0.7    &0.7    &20    &23\\
					0.3     &0.7    &0.8    &20    &23\\
					0.3     &0.7    &0.9    &102   &22\\
					0.3     &0.7    &1      &34    &22\\
					\bottomrule
			\end{tabular}}
			\label{T:log111}
		\end{center}
	\end{minipage}
	\hfill
	\begin{minipage}[!h]{0.50\hsize}\centering
		\begin{center}\resizebox{0.8\textwidth}{!}{%
				%\caption{Лог работы программы.}
				\begin{tabular}{c@{\hspace{5mm}}c@{\hspace{5mm}}c@{\hspace{5mm}}c@{\hspace{5mm}}c@{\hspace{5mm}}c}
					\toprule
					a        &b      &p      &iters &длина пути \\
					\midrule
					0.4     &0.6    &0      &20    &22\\
					0.4     &0.6    &0.1    &20    &22\\
					0.4     &0.6    &0.2    &20    &22\\
					0.4     &0.6    &0.3    &20    &22\\
					0.4     &0.6    &0.4    &20    &22\\
					0.4     &0.6    &0.5    &20    &22\\
					0.4     &0.6    &0.6    &20    &22\\
					0.4     &0.6    &0.7    &49    &22\\
					0.4     &0.6    &0.8    &20    &22\\
					0.4     &0.6    &0.9    &20    &22\\
					0.4     &0.6    &1      &8     &22\\
					\midrule
					0.5     &0.5    &0      &20    &22\\
					0.5     &0.5    &0.1    &20    &22\\
					0.5     &0.5    &0.2    &20    &22\\
					0.5     &0.5    &0.3    &20    &22\\
					0.5     &0.5    &0.4    &20    &22\\
					0.5     &0.5    &0.5    &20    &22\\
					0.5     &0.5    &0.6    &20    &22\\
					0.5     &0.5    &0.7    &20    &22\\
					0.5     &0.5    &0.8    &85    &22\\
					0.5     &0.5    &0.9    &163   &22\\
					0.5     &0.5    &1      &26    &22\\
					\midrule
					0.6     &0.4    &0      &20    &22\\
					0.6     &0.4    &0.1    &20    &23\\
					0.6     &0.4    &0.2    &20    &22\\
					0.6     &0.4    &0.3    &20    &22\\
					0.6     &0.4    &0.4    &20    &22\\
					0.6     &0.4    &0.5    &20    &22\\
					0.6     &0.4    &0.6    &20    &22\\
					0.6     &0.4    &0.7    &124   &23\\
					0.6     &0.4    &0.8    &49    &23\\
					0.6     &0.4    &0.9    &14    &22\\
					0.6     &0.4    &1      &65    &22\\
					\midrule
					0.7     &0.3    &0      &20    &22\\
					0.7     &0.3    &0.1    &20    &22\\
					0.7     &0.3    &0.2    &20    &22\\
					0.7     &0.3    &0.3    &20    &22\\
					0.7     &0.3    &0.4    &20    &22\\
					0.7     &0.3    &0.5    &20    &22\\
					0.7     &0.3    &0.6    &20    &22\\
					0.7     &0.3    &0.7    &20    &22\\
					0.7     &0.3    &0.8    &13    &22\\
					0.7     &0.3    &0.9    &16    &22\\
					0.7     &0.3    &1      &7     &22\\
					\bottomrule
			\end{tabular}}
			%\label{T:log}
		\end{center}
	\end{minipage}
\end{table}
\begin{table}[!h]
	\begin{center}
		\begin{tabular}{c@{\hspace{7mm}}c@{\hspace{7mm}}c@{\hspace{7mm}}c@{\hspace{7mm}}c@{\hspace{7mm}}c}
			\toprule
			a        &b      &p      &iters &длина пути \\
			\midrule
			0.4     &0.6    &0      &20    &22\\
			0.4     &0.6    &0.1    &20    &22\\
			0.4     &0.6    &0.2    &20    &22\\
			0.4     &0.6    &0.3    &20    &22\\
			0.4     &0.6    &0.4    &20    &22\\
			0.4     &0.6    &0.5    &20    &22\\
			0.4     &0.6    &0.6    &20    &22\\
			0.4     &0.6    &0.7    &49    &22\\
			0.4     &0.6    &0.8    &20    &22\\
			0.4     &0.6    &0.9    &20    &22\\
			0.4     &0.6    &1      &8     &22\\
			\midrule
			0.5     &0.5    &0      &20    &22\\
			0.5     &0.5    &0.1    &20    &22\\
			0.5     &0.5    &0.2    &20    &22\\
			0.5     &0.5    &0.3    &20    &22\\
			0.5     &0.5    &0.4    &20    &22\\
			0.5     &0.5    &0.5    &20    &22\\
			0.5     &0.5    &0.6    &20    &22\\
			0.5     &0.5    &0.7    &20    &22\\
			0.5     &0.5    &0.8    &85    &22\\
			0.5     &0.5    &0.9    &163   &22\\
			0.5     &0.5    &1      &26    &22\\
			\midrule
			0.6     &0.4    &0      &20    &22\\
			0.6     &0.4    &0.1    &20    &22\\
			0.6     &0.4    &0.2    &20    &22\\
			0.6     &0.4    &0.3    &20    &22\\
			0.6     &0.4    &0.4    &20    &22\\
			0.6     &0.4    &0.5    &20    &22\\
			0.6     &0.4    &0.6    &20    &22\\
			0.6     &0.4    &0.7    &124   &22\\
			0.6     &0.4    &0.8    &49    &22\\
			0.6     &0.4    &0.9    &14    &22\\
			0.6     &0.4    &1      &65    &22\\
			\bottomrule
		\end{tabular}
	\end{center}
\end{table}

\subsection{Класс данных 2}

Матрица смежности для класса данных 2:
\begin{equation*}
	G = \begin{pmatrix}
		0,      &13220  &5777   &10272  &2509   &12737  &11202  &13053  &2014   &3140   \\
		13220,  &0      &9305   &8955   &3974   &12863  &4135   &509    &13839  &2603   \\
		5777,   &9305   &0      &10978  &5521   &9191   &13678  &3453   &6444   &13320  \\
		10272,  &8955   &10978  &0      &13342  &10270  &8814   &14032  &1896   &6665   \\
		2509,   &3974   &5521   &13342  &0      &6897   &3215   &1483   &11523  &6752   \\
		12737,  &12863  &9191   &10270  &6897   &0      &9091   &5338   &9966   &6815   \\
		11202,  &4135   &13678  &8814   &3215   &9091   &0      &3973   &6879   &10087  \\
		13053,  &509    &3453   &14032  &1483   &5338   &3973   &0      &5463   &8252   \\
		2014,   &13839  &6444   &1896   &11523  &9966   &6879   &5463   &0      &4997   \\
		3140,   &2603   &13320  &6665   &6752   &6815   &10087  &8252   &4997   &0      \\
	\end{pmatrix}
\end{equation*}

В таблице \ref{T:log222} приведены результаты параметризаций метода решения задачи коммивояжера на основании муравьиного алгортима.
Полный перебор определил оптимальную длину пути 22.

\begin{table}[!h]
	\caption{Таблица коэффициентов для класса данных №2}
	\begin{center}\resizebox{0.575\textwidth}{!}{%
			\begin{tabular}{c@{\hspace{7mm}}c@{\hspace{7mm}}c@{\hspace{7mm}}c@{\hspace{7mm}}c@{\hspace{7mm}}c}
				\toprule
				a        &b      &p      &iters &длина пути \\
				\midrule
				0       &1      &0      &12    &40402\\
				0       &1      &0.1    &12    &40402\\
				0       &1      &0.2    &12    &40402\\
				0       &1      &0.3    &62    &40402\\
				0       &1      &0.4    &62    &40402\\
				0       &1      &0.5    &62    &40402\\
				0       &1      &0.6    &62    &40402\\
				0       &1      &0.7    &62    &40402\\
				0       &1      &0.8    &62    &40402\\
				0       &1      &0.9    &62    &40402\\
				0       &1      &1      &62    &40402\\
				\midrule
				0.1     &0.9    &0      &62    &40402\\
				0.1     &0.9    &0.1    &62    &40402\\
				0.1     &0.9    &0.2    &62    &40402\\
				0.1     &0.9    &0.3    &62    &40402\\
				0.1     &0.9    &0.4    &62    &40402\\
				0.1     &0.9    &0.5    &62    &40402\\
				0.1     &0.9    &0.6    &62    &40402\\
				0.1     &0.9    &0.7    &62    &40402\\
				0.1     &0.9    &0.8    &62    &40402\\
				0.1     &0.9    &0.9    &62    &40402\\
				0.1     &0.9    &1      &41    &40402\\
				\midrule
				0.2     &0.8    &0      &62    &40402\\
				0.2     &0.8    &0.1    &62    &40402\\
				0.2     &0.8    &0.2    &62    &40402\\
				0.2     &0.8    &0.3    &62    &40402\\
				0.2     &0.8    &0.4    &62    &40402\\
				0.2     &0.8    &0.5    &92    &40402\\
				0.2     &0.8    &0.6    &62    &40402\\
				0.2     &0.8    &0.7    &62    &40402\\
				0.2     &0.8    &0.8    &10    &40402\\
				0.2     &0.8    &0.9    &10    &40402\\
				0.2     &0.8    &1      &10    &40402\\
				\bottomrule
		\end{tabular}}
		\label{T:log222}
	\end{center}
\end{table}
\begin{table}
	\begin{minipage}[!h]{0.10\hsize}\centering
		\begin{center}\resizebox{4\textwidth}{!}{%
				\begin{tabular}{c@{\hspace{5mm}}c@{\hspace{5mm}}c@{\hspace{5mm}}c@{\hspace{5mm}}c@{\hspace{5mm}}c}
					\toprule
					a        &b      &p      &iters &длина пути \\
					\midrule
					0.3     &0.7    &0      &28    &40402\\
					0.3     &0.7    &0.1    &28    &40402\\
					0.3     &0.7    &0.2    &206   &40402\\
					0.3     &0.7    &0.3    &102   &40402\\
					0.3     &0.7    &0.4    &28    &40402\\
					0.3     &0.7    &0.5    &10    &40402\\
					0.3     &0.7    &0.6    &10    &40402\\
					0.3     &0.7    &0.7    &10    &40402\\
					0.3     &0.7    &0.8    &10    &40402\\
					0.3     &0.7    &0.9    &10    &40402\\
					0.3     &0.7    &1      &10    &40402\\
					\midrule
					0.4     &0.6    &0      &28    &40402\\
					0.4     &0.6    &0.1    &28    &40402\\
					0.4     &0.6    &0.2    &206   &40402\\
					0.4     &0.6    &0.3    &206   &40402\\
					0.4     &0.6    &0.4    &28    &40402\\
					0.4     &0.6    &0.5    &28    &40402\\
					0.4     &0.6    &0.6    &10    &40402\\
					0.4     &0.6    &0.7    &10    &40402\\
					0.4     &0.6    &0.8    &10    &40402\\
					0.4     &0.6    &0.9    &10    &40402\\
					0.4     &0.6    &1      &10    &40402\\
					\midrule
					0.5     &0.5    &0      &28    &40402\\
					0.5     &0.5    &0.1    &28    &40402\\
					0.5     &0.5    &0.2    &28    &40402\\
					0.5     &0.5    &0.3    &28    &40402\\
					0.5     &0.5    &0.4    &54    &40402\\
					0.5     &0.5    &0.5    &10    &40402\\
					0.5     &0.5    &0.6    &10    &40402\\
					0.5     &0.5    &0.7    &10    &40402\\
					0.5     &0.5    &0.8    &10    &40402\\
					0.5     &0.5    &0.9    &10    &40402\\
					0.5     &0.5    &1      &10    &40402\\
					\midrule
					0.6     &0.4    &0      &28    &40402\\
					0.6     &0.4    &0.1    &28    &40402\\
					0.6     &0.4    &0.2    &28    &40402\\
					0.6     &0.4    &0.3    &54    &40402\\
					0.6     &0.4    &0.4    &127   &40402\\
					0.6     &0.4    &0.5    &89    &40402\\
					0.6     &0.4    &0.6    &26    &40402\\
					0.6     &0.4    &0.7    &150   &40402\\
					0.6     &0.4    &0.8    &139   &40402\\
					0.6     &0.4    &0.9    &193   &40402\\
					0.6     &0.4    &1      &63    &40402\\
					\bottomrule
			\end{tabular}}
		\end{center}
	\end{minipage}
	\hfill
	\begin{minipage}[!h]{0.50\hsize}\centering
		\begin{center}\resizebox{0.8\textwidth}{!}{%
				\begin{tabular}{c@{\hspace{5mm}}c@{\hspace{5mm}}c@{\hspace{5mm}}c@{\hspace{5mm}}c@{\hspace{5mm}}c}
					\toprule
					a        &b      &p      &iters &длина пути \\
					\midrule
					0.7     &0.3    &0      &193   &40402\\
					0.7     &0.3    &0.1    &240   &40402\\
					0.7     &0.3    &0.2    &192   &40402\\
					0.7     &0.3    &0.3    &193   &40402\\
					0.7     &0.3    &0.4    &193   &40402\\
					0.7     &0.3    &0.5    &35    &40402\\
					0.7     &0.3    &0.6    &35    &40402\\
					0.7     &0.3    &0.7    &60    &40402\\
					0.7     &0.3    &0.8    &35    &40402\\
					0.7     &0.3    &0.9    &80    &40402\\
					0.7     &0.3    &1      &35    &40402\\
					\midrule
					0.8     &0.2    &0      &35    &40402\\
					0.8     &0.2    &0.1    &58    &40402\\
					0.8     &0.2    &0.2    &35    &40402\\
					0.8     &0.2    &0.3    &60    &40402\\
					0.8     &0.2    &0.4    &60    &40402\\
					0.8     &0.2    &0.5    &58    &40402\\
					0.8     &0.2    &0.6    &60    &40402\\
					0.8     &0.2    &0.7    &58    &40402\\
					0.8     &0.2    &0.8    &35    &40402\\
					0.8     &0.2    &0.9    &58    &40402\\
					0.8     &0.2    &1      &96    &40402\\
					\midrule
					0.9     &0.1    &0      &58    &40402\\
					0.9     &0.1    &0.1    &35    &40402\\
					0.9     &0.1    &0.2    &58    &40402\\
					0.9     &0.1    &0.3    &60    &40402\\
					0.9     &0.1    &0.4    &80    &40402\\
					0.9     &0.1    &0.5    &80    &40402\\
					0.9     &0.1    &0.6    &60    &40402\\
					0.9     &0.1    &0.7    &41    &40402\\
					0.9     &0.1    &0.8    &3     &40402\\
					0.9     &0.1    &0.9    &3     &40402\\
					0.9     &0.1    &1      &3     &40402\\
					\midrule
					1       &0   &0      &41    &40402\\
					1       &0   &0.1    &41    &40402\\
					1       &0   &0.2    &41    &40402\\
					1       &0   &0.3    &41    &40402\\
					1       &0   &0.4    &41    &40402\\
					1       &0   &0.5    &41    &40402\\
					1       &0   &0.6    &41    &40402\\
					1       &0   &0.7    &28    &40402\\
					1       &0   &0.8    &41    &40402\\
					1       &0   &0.9    &41    &40402\\
					1       &0   &1      &64    &40402\\
					\bottomrule
			\end{tabular}}
		\end{center}
	\end{minipage}
\end{table}

\section*{Вывод}

Таким образом, на основе полученных таблиц можно сделать вывод, что при классе данных, содержащем приблизительно равные значения наилучшими наборами стали ($\alpha = 0.2, \beta = 0.8, P = 1$), при данных значениях алгоритм нашел лучший путь за 2 запуска.
При наборах  ($\alpha = 0, \beta = 1, P = 1$), ($\alpha = 0.7, \beta = 0.3, P = 1$) алгоритм нашел путь за 7 итераций.

Для второго класса данных было определено, что при ($\alpha = 0.9, \beta = 0.1, P = 0.8$), ($\alpha = 0.9, \beta = 0.1, P = 0.9$), ($\alpha = 0.9, \beta = 0.1, P = 1$) алгоритм отработал за 3 итерации.


\chapter*{Заключение}
\addcontentsline{toc}{chapter}{Заключение}

Таким образом, в ходе лабораторной работы было сделано следующее:
\begin{itemize}
	\item дана постановку задачи;
	\item описаны методы полного перебора и эвристический, основанный на муравьином алгоритме;
	\item реализованы данные методы;
	\item выбраны и подготовлены классы данных;
	\item проведена параметризация метода, основанного на муравьином алгоритме;
\end{itemize}

Были также сделаны выводы на основе полученных данных.
Эвристический метод, основанный на муравьином алгоритме имеет преимущество перед методом полного перебора за счет того, что способен работать с данными достаточно большого объема, в то время как полный перебор сильно ограничен размером данных.
Также были подобраны параметры для оптимальной работы метода на двух классах данных.
Однако, в отличии от полного перебора, эвристический алгоритм не гарантирует точность найденного им пути, есть вероятность, что путь будет не оптимален.


\addcontentsline{toc}{chapter}{Литература}
\bibliographystyle{utf8gost705u}  % стилевой файл для оформления по ГОСТу
\bibliography{51-biblio}          % имя библиографической базы (bib-файла)


\end{document}
