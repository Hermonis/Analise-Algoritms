\documentclass[a4paper,14pt]{extreport}

\usepackage{cmap} % Улучшенный поиск русских слов в полученном pdf-файле
\usepackage[T2A]{fontenc} % Поддержка русских букв
\usepackage[utf8]{inputenc} % Кодировка utf8
\usepackage[english,russian]{babel} % Языки: русский, английский
%\usepackage{pscyr} % Нормальные шрифты

\usepackage{amsmath}

\usepackage{geometry}
\geometry{left=30mm}
\geometry{right=15mm}
\geometry{top=20mm}
\geometry{bottom=20mm}

\usepackage{titlesec}
\titleformat{\section}
{\normalsize\bfseries}
{\thesection}
{1em}{}
\titlespacing*{\chapter}{0pt}{-30pt}{8pt}
\titlespacing*{\section}{\parindent}{*4}{*4}
\titlespacing*{\subsection}{\parindent}{*4}{*4}

\usepackage{setspace}
\onehalfspacing % Полуторный интервал

\frenchspacing
\usepackage{indentfirst} % Красная строка

\usepackage{titlesec}
\titleformat{\chapter}{\LARGE\bfseries}{\thechapter}{20pt}{\LARGE\bfseries}
\titleformat{\section}{\Large\bfseries}{\thesection}{20pt}{\Large\bfseries}

\usepackage{listings}
\usepackage{xcolor}


\lstdefinestyle{go}{
	language=Go,
	backgroundcolor=\color{white},
	basicstyle=\footnotesize\ttfamily,
	keywordstyle=\color{blue},
	stringstyle=\color{red},
	commentstyle=\color{gray},
	stepnumber=1,
	numbersep=5pt,
	frame=single,
	tabsize=8,
	captionpos=b,
	breaklines=true,
	breakatwhitespace=true,
	escapeinside={\#*}{*)},
	morecomment=[l][\color{magenta}]{\#},
	columns=fullflexible
}

\usepackage{pgfplots}
\usetikzlibrary{datavisualization}
\usetikzlibrary{datavisualization.formats.functions}

\usepackage{graphicx}
\newcommand{\img}[3] {
	\begin{figure}[h]
		\center{\includegraphics[height=#1]{inc/img/#2}}
		\caption{#3}
		\label{img:#2}
	\end{figure}
}
\newcommand{\boximg}[3] {
	\begin{figure}[h]
		\center{\fbox{\includegraphics[height=#1]{inc/img/#2}}}
		\caption{#3}
		\label{img:#2}
	\end{figure}
}
\newcommand{\imgext}[4] {
	\begin{figure}[h!]
		\center{\includegraphics[#1]{inc/img/#2.#3}}
		\caption{#4}
		\label{img:#2}
	\end{figure}
}

\usepackage[justification=centering]{caption} % Настройка подписей float объектов

\usepackage[unicode,pdftex]{hyperref} % Ссылки в pdf
\hypersetup{hidelinks}

\newcommand{\code}[1]{\texttt{#1}}

\usepackage{icomma} % Интеллектуальные запятые для десятичных чисел

\usepackage{csvsimple}

\usepackage{svg}


\begin{document}

\begin{titlepage}
	\centering
	
	{\footnotesize\itshape Министерство науки и высшего образования
		Российской Федерации Федеральное государственное бюджетное
		образовательное учреждение высшего образования «Московский
		государственный технический университет имени Н.~Э.~Баумана
		(национальный исследовательский университет)» (МГТУ им. Н.~Э.~Баумана)
		\\}
	
	\vspace{60mm}
	
	\textbf{ОТЧЕТ}\\
	По лабораторной работе №5\\
	По курсу: «Анализ алгоритмов»\\
	Тема: «Конвейер»\\
	
	\vspace{60mm}
	
	\hspace{70mm} Студент:\hfill Козаченко~А.~А.\\
	\hspace{70mm} Группа: \hfill ИУ7-54Б\\
	%\vspace{5mm}
	\hspace{70mm} Преподаватели:\hfill Волкова~Л.~Л.,\\
	\hfill Строганов~Ю.~В.\\
	
	\vfill
	Москва, 2019
\end{titlepage}

\tableofcontents

\chapter*{Введение}
\addcontentsline{toc}{chapter}{Введение}

При обработке данных могут возникать ситуации, когда необходимо обработать множество данных последовательно несколькими алгоритмами. В этом случае удобно использовать конвейерную обработку данных. Это может быть полезно при следующих задачах:
\begin{itemize}
	\item шифровании данных;
	\item сортировки и фильтрации данных;
	\item и др.
\end{itemize}

Цель данной работы: получить навык организации асинхронного взаимодействия потоков на примере конвейерной обработки данных.

В рамках выполнения работы необходимо решить следующие задачи:
\begin{itemize}
	\item рассмотреть и изучить конвейерную обработку данных;
	\item реализовать конвейер с количеством лент не меньше трех в многопоточной среде;
	\item на основании проделанной работы сделать выводы.
\end{itemize}


\chapter{Аналитическая часть}

\section{Описание конвейерной обработки данных}

При конвейерной обработке данных каждая лента имеет свою очередь с некоторыми задачами, ожидающими обработки. Лента берет данные из своей очереди с входными данными, проводит с ними необходимые операции и передает в очередь следующей ленты или, в случае последней ленты, в пул обработанных задач.

\section*{Вывод}
В данной работе стоит задача реализации конвейера.

\chapter{Конструкторская часть}

\section{Разработка конвейерной обработки данных}

Принцип работы стадийной обработки представлен на рис.~\ref{img:schema}.

\imgext{height=70mm}{schema}{pdf}{Конвейерная обработка данных}

\section*{Вывод}

Был показан принцип работы конвейерной обработки данных.

\chapter{Технологическая часть}

В данном разделе приведены средства реализации и листинг кода.

\section{Средства реализации}

В качестве языка программирования был выбран Go, так как он предоставляет широкие возможности и крайне удобный интерфейс для эффективной реализации асинхронной, параллельной обработки данных \cite{go}.

Для измерения времени использовалась стандартная библиотека \code{time}.
Так как основное время работы составляет ожидание \code{sleep}, то достаточно замерить время работы один раз.

\section{Листинг кода}

В листинге \ref{lst:conveyor} представлена реализация конвейерная обработка данных.

\begin{lstinputlisting}[
	caption={Алгоритм сортировки пузырьком},
	label={lst:conveyor},
	style={go}
	]{../main.go}
\end{lstinputlisting}

\section{Тестирование функций}

Для тестирования были реализованы функции, представленные в листинге \ref{lst:test}.
Результаты тестирования представлены в таблице~\ref{tbl:test}. Видно, что тестирование пройдено успешно.

\begin{lstinputlisting}[
	caption={Тестовые задачи},
	label={lst:test},
	style={go},
	linerange={12-27}
	]{../main_test.go}
\end{lstinputlisting}

\begin{table}[h!]
	\begin{center}
		\begin{tabular}{|c|c|c|}
			\hline
			Входные данные & Ожидаемый результат & Результат \\ 
			\hline
			1,3,4 & 24  & 24\\
			\hline
		\end{tabular}
		\caption{\label{tbl:test}Тестирование конвейерной обработки}
	\end{center}
\end{table}

\section*{Вывод}

Правильный выбор инструментов разработки позволил эффективно реализовать алгоритмы, настроить модульное тестирование и выполнить исследовательский раздел лабораторной работы.


\chapter{Исследовательская часть}

\section{Время выполнения}

Реализованная контейнерная обработка данных работает за 1,2с. Последовательная реализация потребовала бы $(0,1 + 0,2 + 0,3) * 3 = 1,8\text{с}$, что на 50\% медленней.

\section*{Вывод}

Алгоритм сортировки вставками работает лучше остальных двух на случайных числах и уже отсортированных, практический интерес, конечно, представляет лишь первый случай,
на котором сортировка вставками почти на четверть быстрее сортировки пузырьком и на тринадцатую часть быстрее сортировки выбором.


\chapter*{Заключение}
\addcontentsline{toc}{chapter}{Заключение}

В рамках лабораторной работы была рассмотрена и изучена конвейерная обработка данных.
Благодаря ней возможна крайне удобная реализация задач, требующих поэтапной обработки некоторого набора данных, а также в некоторых случаях позволяет значительно ускорить выполнение программы (в реализованном синтетическом примере выигрыш составил 50\%).

\addcontentsline{toc}{chapter}{Литература}
\bibliographystyle{utf8gost705u}  % стилевой файл для оформления по ГОСТу
\bibliography{51-biblio}          % имя библиографической базы (bib-файла)


\end{document}
