\documentclass[12pt]{report}
\usepackage[utf8]{inputenc}
\usepackage[russian]{babel}
%\usepackage[14pt]{extsizes}
\usepackage{listings}

\usepackage{graphicx}
\graphicspath{{src/}}
\DeclareGraphicsExtensions{.pdf,.png,.jpg}

% Для листинга кода:
\lstset{ %
	language=python,                 % выбор языка для подсветки (здесь это python)
	basicstyle=\small\sffamily, % размер и начертание шрифта для подсветки кода
	numbers=left,               % где поставить нумерацию строк (слева\справа)
	numberstyle=\tiny,           % размер шрифта для номеров строк
	stepnumber=1,                   % размер шага между двумя номерами строк
	numbersep=5pt,                % как далеко отстоят номера строк от подсвечиваемого кода
	showspaces=false,            % показывать или нет пробелы специальными отступами
	showstringspaces=false,      % показывать или нет пробелы в строках
	showtabs=false,             % показывать или нет табуляцию в строках
	frame=single,              % рисовать рамку вокруг кода
	tabsize=2,                 % размер табуляции по умолчанию равен 2 пробелам
	captionpos=t,              % позиция заголовка вверху [t] или внизу [b] 
	breaklines=true,           % автоматически переносить строки (да\нет)
	breakatwhitespace=false, % переносить строки только если есть пробел
	escapeinside={\#*}{*)}   % если нужно добавить комментарии в коде
}

% Для измененных титулов глав:
\usepackage{titlesec, blindtext, color} % подключаем нужные пакеты
\definecolor{gray75}{gray}{0.75} % определяем цвет
\newcommand{\hsp}{\hspace{20pt}} % длина линии в 20pt
% titleformat определяет стиль
\titleformat{\chapter}[hang]{\Huge\bfseries}{\thechapter\hsp\textcolor{gray75}{|}\hsp}{0pt}{\Huge\bfseries}


% plot
\usepackage{pgfplots}
\usepackage{filecontents}
\usetikzlibrary{datavisualization}
\usetikzlibrary{datavisualization.formats.functions}


\begin{document}
	%\def\chaptername{} % убирает "Глава"
	\begin{titlepage}
		\centering
		{\scshape\LARGE МГТУ им. Баумана \par}
		\vspace{3cm}
		{\scshape\Large Лабораторная работа №3\par}
		\vspace{0.5cm}	
		{\scshape\Large По курсу: "Анализ алгоритмов"\par}
		\vspace{1.5cm}
		{\huge\bfseries Алгоритмы сортировки\par}
		\vspace{2cm}
		\Large Работу выполнила: Козаченко Александр, ИУ7-54Б\par
		\vspace{0.5cm}
		\LargeПреподаватели:  Волкова Л.Л., Строганов Ю.В.\par
		
		\vfill
		\large \textit {Москва, 2019} \par
	\end{titlepage}
	
	\tableofcontents
	
	\newpage
	\chapter*{Введение}
	\addcontentsline{toc}{chapter}{Введение}
	
	На данный момент существует огромное количество вариаций сортировок.
	Эти алгоритмы необходимо уметь сравнивать, чтобы выбирать наилучше подходящие в конкретном случае. 
	
	Эти алгоримы оцениваются по:
	
	\begin{itemize}
		\item Времени быстродействия
		\item Затратам памяти
	\end{itemize}
	
	Целью данной лабораторной работы является изучение применений алгоритмов сортировки, обучение расчету трудоемкости алгоритмов.
	
	
	\chapter{Аналитическая часть}
	\section{Описание алгоритмов}
	Сортировка массива — одна из самых популярных операций над массивом. Алгоритмы реализуют упорядочивание элементов в списке.
	В случае, когда элемент списка имеет несколько полей, поле, служащее критерием порядка, называется ключом сортировки
	\textbf{Область применения:} 
	\begin{itemize}
		\item физика,
		\item математика,
		\item экономика,
		\item итд.
	\end{itemize}
	
	\subsection{Сортировка вставками}
	На каждом шаге выбирается один из элементов неотсортированной части массива (максимальный/минимальный) 
	и помещается на нужную позицию в отсортированную часть массива.
	
	\subsection{Гномья сортировка}
	Это метод, которым садовый гном сортирует линию цветочных горшков. По существу он смотрит на текущий и предыдущий садовые горшки: если они в правильном порядке, он шагает на один горшок вперёд, иначе он меняет их местами и шагает на один горшок назад. Граничные условия: если нет предыдущего горшка, он шагает вперёд; если нет следующего горшка, он закончил.
	
	\subsection {Быстрая сортировка}
	Массив разбивается на два (возможно пустых) подмассива. Таких, что в одном подмассиве каждый элемент меньше либо равен опорному, 
	и при этом не превышает любой элемент второго подмассива. Опорный элемент вычислыется в ходе процедуры разбиения. 
	Подмассивы сортируются с помощью рекурсивного вызова процедуры быстрой сортировки. 
	Поскольку подмассивы сортируются на месте, для их объединения не трубуются никакие действия.
	
	
	\chapter{Конструкторская часть}
	\section{Схемы алгоритмов}
	В данном разделе будут рассмотрены схемы алгоритмов гномьей сортировки (\ref{ris:imageSB}), сортировки вставками (\ref{ris:imageSI}), быстрой сортировки (\ref{ris:imageSQ}).
	
	\newpage
	\begin{figure}[h]
		\includegraphics[width=1\linewidth]{gnome.png}
		\caption{Схема алгоритма гномьей сортировки}
		\label{ris:imageSB}
	\end{figure}
	
	\newpage
	\begin{figure}[h]
		\includegraphics[width=1\linewidth]{insert.png}
		\caption{Схема алгоритма сортировки вставками}
		\label{ris:imageSI}
	\end{figure}
	
	\newpage
	\begin{figure}[h]
		\includegraphics[scale=0.5]{quick.png}
		\caption{Схема алгоритма быстрой сортировки}
		\label{ris:imageSQ}
	\end{figure}
	
	\newpage
	\begin{figure}[h]
		\includegraphics[scale=0.5]{partition.png}
		\caption{Схема функции partition}
		\label{ris:imageSP}
	\end{figure}
	
	
	\section{Трудоемкость алгоритмов}
	Введем модель трудоемкости для оценки алгоритмов:
	\begin{enumerate}
		\item  базовые операции стоимостью 1 — +, -, *, /, =, ==, <=, >=, !=, +=, [], получение полей класса
		\item оценка трудоемкости цикла: Fц = a + N*(a + Fтела), где a - условие цикла
		\item стоимость условного перехода возьмем за 0, стоимость вычисления условия остаётся.
	\end{enumerate}
	
	Далее будет приведены оценки трудоемкости алгоритмов. Построчная оценка трудоемкости гномьей сортировки (Табл. 2.1).
	
	\subsection{Гномья сортировка}
	\begin{center}
		Табл. 2.1 Построчная оценка веса
		\begin{tabular}{|l c|} 
			\hline
			Код & Вес \\ [0.5ex] 
			\hline\hline
			int i = 0; & 1\\
			\hline
			while (i < len(arr)) & 2\\
			\hline
			if not i or arr[i - 1] <= arr[i]: & 2\\
			\hline
			else : & 1\\
			\hline
			arr[i], arr[i - 1] = arr[i - 1] , arr[i] & 2 + 3\\
			\hline
			i -= 1 & 1\\
			\hline
		\end{tabular}
	\end{center}
	
	
	\textbf{Лучший случай:} Массив отсортирован; не произошло ни одного обмена за 1 проход -> выходим из цикла \newline
	Трудоемкость:  $1 + 1 + n* (2 + 7 + 1 + 3 ) + 2 = 13n + 4 = O(n)$
	
	\textbf{Худший случай:}  Массив отсортирован в обратном порядке; в каждом случае происходил обмен\newline
	Трудоемкость: $1 + 1 +  n*(n* (7 + 5 + 1  + 3)  + 1 + 1) + 2 =  n*(16n + 2) + 4 = 16n^2 + 2n + 4 = O(n^2)$
	
	
	\subsection{Сортировка вставками}
	\hspace*{5mm}
	\textbf{Лучший случай:} отсортированный массив. При этом все внутренние циклы состоят всего из одной итерации.\newline
	Трудоемкость: $T(n) = 3n + ((2 + 2 + 4 + 2 ) * (n - 1))  =  3n + 10(n-1) = 13n - 10 = O(n)$
	
	\textbf{Худший случай:} массив отсортирован в обратном нужному порядке. Каждый новый элемент сравнивается со всеми в отсортированной последовательности.
	Все внутренние циклы будут состоять из j итераций. \newline
	Трудоемкость: $T(n) = 3n + (2 + 2)(n-1) + 4 \left({\frac {n(n+1)}{2}}-1\right)+5{\frac {n(n-1)}{2}}+3(n-1) = 3n + 4n - 4 + 2n^2
	+ 2n - 4 + 2.5n^2 - 2.5n + 3n - 3 = 4.5n^2 + 9.5n - 11 = O(n^{2})$
	
	\subsection{Быстрая сортировка}
	\hspace*{5mm}
	\textbf{Лучший случай:} сбалансированное дерево вызовов \(O(n*log(n))\)  
	В наиболее благоприятном случае процедура PARTITION приводит к двум подзадачам, размер каждой из которых не превышает $\frac{n}{2}$, поскольку размер одной из них равен $\frac{n}{2}$ , а второй$\frac{n}{2} - 1$. В такой ситуации быстрая сортировка работает намного производительнее, и время ее работы описывается следующим рекуррентным соотношением: $T(n) = 2T(\frac{n}{2}) + O(n)$,где мы не обращаем внимания на неточность, связанную с игнорированием функций “пол” и “потолок”, и вычитанием 1. Это рекуррентное соотношение имеет решение ; $T(n) =O(nlgn)$. При сбалансированности двух частей разбиения на каждом уровне рекурсии мы получаем асимптотически более быстрый алгоритм.
	
	Фактически любое разбиение, характеризующееся конечной константой пропорциональности, приводит к образованию дерева рекурсии высотой $O(lgn)$ со стоимостью каждого уровня, равной $O(n)$. Следовательно, прилюбой постоянной пропорции разбиения полное время работы быстрой сортировки составляет $O(nlgn)$.
	
	\textbf{Худший случай:} несбалансированное дерево $O(n^2)$
	Поскольку рекурсивный вызов процедуры разбиения, на вход которой подается массив размером 0, приводит к немедленному возврату из этой процедуры без выполнения каких-ли-бо операций, $T(0) = O(1)$. Таким образом, рекуррентное соотношение, описывающее время работы процедуры в указанном случае, записывается следующим образом: 
	$T(n) =T(n-1) +T(0) + O(n) =T(n-1) + O(n)$. Интуитивно понятно, что при суммировании промежутков времени, затрачиваемых на каждый уровень рекурсии, получается арифметическая прогрессия, что приводит к результату $O(n^2)$.
	
	\section{Вывод}
	Гномья сортировка: лучший - $O(n)$, худший - $O(n^2)$ \newline
	Сортировка вставками: лучший - $O(n)$, худший - $O(n^2)$ \newline
	Быстрая сортировка: лучший - $O(nlgn)$, худший - $O(n^2)$ \newline
	
	При этом сортировка вставками быстрее гномьей с флагом в худшем случае т.к. имеет меньший коэффициент. Вставки $4.5n^2$, гномья сортировка $16n^2$.
	
	
	\chapter{Технологическая часть}
	\section{Выбор ЯП}
	В качестве языка программирования был выбран C\#, а средой разработки Visual Studio.
	Время работы алгоритмов было замерено с помощью класса Stopwatch.
	
	\section{Описание структуры ПО}
	Ниже представлена IDEF0 диаграмма (\ref{ris:imageIDEF}), описывающая структуру программы.
	\begin{figure}[h]
		\caption{Функциональная схема сортировки массива (IDEF0 диаграмма 1 уровня)}
		\label{ris:imageIDEF}
	\end{figure}
	
	\section{Сведения о модулях программы}
	Программа состоит из:
	\begin{itemize}
		\item Program.cs - главный файл программы, в котором располагается точка входа в программу и функция замера времени.
		\item Sortiong.cs - файл класса sorting, в котором находятся алгоримы сортировки
	\end{itemize}
	
	\section{Листинг кода}
	\begin{lstlisting}[label=gnome,caption=Алгоритм гномьей сортировки]
		def gnome_sort(arr):
			i = 1
			while i < len(arr):
				if not i or arr[i - 1] <= arr[i]:
					i += 1
			else :
				arr[i], arr[i - 1] = arr[i - 1] , arr[i]
				i -= 1
			return arr
	\end{lstlisting}
	\newpage
	\begin{lstlisting}[label=insertion,caption=Алгоритм сортировки вставками]
		def insertion_sort(arr):
			for i in range(1, len(arr)):
				key = arr[i]
				j = i-1
				while j >=0 and key < arr[j] :
					arr[j+1] = arr[j]
					j -= 1
				arr[j+1] = key
			return arr
	\end{lstlisting}
	
	\begin{lstlisting}[label=quick,caption=Алгоритм быстрой сортировки]
		def quick_sort(nums):  
			def _quick_sort(items, low, high):
				if low < high:
					split_index = partition(items, low, high)
					_quick_sort(items, low, split_index)
					_quick_sort(items, split_index + 1, high)
			_quick_sort(nums, 0, len(nums) - 1)
	\end{lstlisting}
	
	\begin{lstlisting}[label=pivot,caption=Алгоритм поиска опорного элемента]
		def partition(nums, low, high):
			pivot = nums[(low + high) // 2]
			i = low - 1
			j = high + 1
			while True:
				i += 1
				while nums[i] < pivot:
					i += 1
				j -= 1
				while nums[j] > pivot:
					j -= 1
				if i >= j:
					return j
				nums[i], nums[j] = nums[j], nums[i]
	\end{lstlisting}
	
	
	\section{Вывод}
	В данном разделе были представлены реализации алгоритмов гномьей сортировки, сортировки вставками и быстрой сортировки.
	
	\chapter{Исследовательская часть}
	Был проведен замер времени работы каждого из алгоритмов.
	
	\section{Примеры работы программы}
	\begin{figure}[h]
		\center{\includegraphics[scale=0.3]{Test1}} 
		\caption{Сортировка массива, заполненного случайными числами}
		\label{ris:image}
	\end{figure}
	\newpage
	\begin{figure}[h]
		\center{\includegraphics[scale=0.3]{Test2}} 
		\caption{Сортировка массива, заполненного случайными числами}
		\label{ris:image}
	\end{figure}
	
	\begin{figure}[h]
		\center{\includegraphics[scale=0.3]{Test3}} 
		\caption{Сортировка массива, заполненного одинаковыми числами}
		\label{ris:image}
	\end{figure}
	\newpage
	\begin{figure}[h]
		\center{\includegraphics[scale=0.3]{Test4}} 
		\caption{Сортировка пустого массива}
		\label{ris:image}
	\end{figure}
	
	\section{Эксперимент}
	В рамках данного эксперимента было произведено сравнение времени выполнения трех алгоритмов в лучшем/худшем/случайном
	случае заполения массива. При длине массивов от 100 до 1000 элементов с шагом 100. На предоставленных ниже графиках (Рис. 4.1), (Рис. 4.2), (Рис. 4.3)
	по оси l идет длина массива, а по оси t - время сортировки в тиках.
	
	\begin{tikzpicture}
		\begin{axis}[
			axis lines = left,
			xlabel = $l$,
			ylabel = {$t$},
			legend pos=north west,
			ymajorgrids=true
			]
			\addplot[color=red, mark=square] table[x index=0, y index= 1] {BubbleAsc.txt}; 
			\addplot[color=blue, mark=square] table[x index=0, y index= 1] {InsertionAsc.txt}; 
			\addplot[color=green, mark=square] table[x index=0, y index= 1] {QuickAsc.txt}; 
			
			\addlegendentry{GnomeSort}
			\addlegendentry{InsertionSort}
			\addlegendentry{QuickSort}
		\end{axis}
	\end{tikzpicture}
	\begin{center}
		Pис. 4.1: Сравнение времени в лучшем случае
	\end{center}
	
	
	\begin{tikzpicture}
		\begin{axis}[
			axis lines = left,
			xlabel = $l$,
			ylabel = {$t$},
			legend pos=north west,
			ymajorgrids=true
			]
			\addplot[color=red, mark=square] table[x index=0, y index= 1] {BubbleDec.txt}; 
			\addplot[color=blue, mark=square] table[x index=0, y index= 1] {InsertionDec.txt}; 
			\addplot[color=green, mark=square] table[x index=0, y index= 1] {QuickSame.txt}; 
			
			\addlegendentry{GnomeSort}
			\addlegendentry{InsertionSort}
			\addlegendentry{QuickSort}
		\end{axis}
	\end{tikzpicture}
	\begin{center}
		Pис. 4.2: Сравнение времени в худшем случае
	\end{center}
	
	
	\begin{tikzpicture}
		\begin{axis}[
			axis lines = left,
			xlabel = $l$,
			ylabel = {$t$},
			legend pos=north west,
			ymajorgrids=true
			]
			\addplot[color=red, mark=square] table[x index=0, y index= 1] {BubbleRnd.txt}; 
			\addplot[color=blue, mark=square] table[x index=0, y index= 1] {InsertionRnd.txt}; 
			\addplot[color=green, mark=square] table[x index=0, y index= 1] {QuickRnd.txt}; 
			
			\addlegendentry{GnomeSort}
			\addlegendentry{InsertionSort}
			\addlegendentry{QuickSort}
		\end{axis}
	\end{tikzpicture}
	\begin{center}
		Pис. 4.3: Сравнение времени при случайном заполнении массива
	\end{center}
	
	\section{Вывод}
	По результатам тестирования выявлено, что все рассматриваемые алгоритмы реализованы правильно. Самым быстрым алгоритмом, при использовании случайного заполнения, оказался алгоритм быстрой сортировки, а самым медленным — алгоритм гномьей сортировки.
	
	\chapter*{Заключение}
	\addcontentsline{toc}{chapter}{Заключение}
	В ходе работы были изучены алгоритмы сортировки массива: гномья, вставки, быстрая сортировка. Выполнено сравнение всех рассматриваемых алгоритмов. В ходе исследования был найден оптимальный алгоритм. Изучены зависимости выполнения алгоритмов от длины массива. Также реализован программный код продукта.
	
	
	\chapter*{Список литературы}
	\addcontentsline{toc}{chapter}{Список литературы}
	\begin{enumerate}
		\item Кормен Т. Алгоритмы: построение и анализ [Текст] / Кормен Т. - Вильямс, 2014. - 198 с. - 219 с.
	\end{enumerate}

\end{document}