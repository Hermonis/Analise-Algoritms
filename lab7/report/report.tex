\documentclass[a4paper,14pt]{extreport}

\usepackage{cmap} % Улучшенный поиск русских слов в полученном pdf-файле
\usepackage[T2A]{fontenc} % Поддержка русских букв
\usepackage[utf8]{inputenc} % Кодировка utf8
\usepackage[english,russian]{babel} % Языки: русский, английский
%\usepackage{pscyr} % Нормальные шрифты

\usepackage{amsmath}

\usepackage{geometry}
\geometry{left=30mm}
\geometry{right=15mm}
\geometry{top=20mm}
\geometry{bottom=20mm}

\usepackage{titlesec}
\titleformat{\section}
{\normalsize\bfseries}
{\thesection}
{1em}{}
\titlespacing*{\chapter}{0pt}{-30pt}{8pt}
\titlespacing*{\section}{\parindent}{*4}{*4}
\titlespacing*{\subsection}{\parindent}{*4}{*4}

\usepackage{setspace}
\onehalfspacing % Полуторный интервал

\frenchspacing
\usepackage{indentfirst} % Красная строка

\usepackage{titlesec}
\titleformat{\chapter}{\LARGE\bfseries}{\thechapter}{20pt}{\LARGE\bfseries}
\titleformat{\section}{\Large\bfseries}{\thesection}{20pt}{\Large\bfseries}

\usepackage{listings}
\usepackage[table,xcdraw]{xcolor}


\lstdefinestyle{cpp}{
	language=C++,
	backgroundcolor=\color{white},
	basicstyle=\footnotesize\ttfamily,
	keywordstyle=\color{blue},
	stringstyle=\color{red},
	commentstyle=\color{gray},
	directivestyle=\color{orange}
	numbers=left,
	numberstyle=\tiny,
	stepnumber=1,
	numbersep=5pt,
	frame=single,
	tabsize=8,
	captionpos=b,
	breaklines=true,
	breakatwhitespace=true,
	escapeinside={\#*}{*)},
	morecomment=[l][\color{magenta}]{\#},
	columns=fullflexible
}

\usepackage{pgfplots}
\usetikzlibrary{datavisualization}
\usetikzlibrary{datavisualization.formats.functions}

\usepackage{graphicx}
\newcommand{\img}[3] {
	\begin{figure}[h]
		\center{\includegraphics[height=#1]{inc/img/#2}}
		\caption{#3}
		\label{img:#2}
	\end{figure}
}
\newcommand{\boximg}[3] {
	\begin{figure}[h]
		\center{\fbox{\includegraphics[height=#1]{inc/img/#2}}}
		\caption{#3}
		\label{img:#2}
	\end{figure}
}
\newcommand{\imgext}[4] {
	\begin{figure}[h!]
		\center{\includegraphics[#1]{inc/img/#2.#3}}
		\caption{#4}
		\label{img:#2}
	\end{figure}
}

\usepackage[justification=centering]{caption} % Настройка подписей float объектов

\usepackage[unicode,pdftex]{hyperref} % Ссылки в pdf
\hypersetup{hidelinks}

\newcommand{\code}[1]{\texttt{#1}}

\usepackage{icomma} % Интеллектуальные запятые для десятичных чисел

\usepackage{csvsimple}

\usepackage{svg}


\begin{document}

\tableofcontents

\begin{titlepage}
	\centering
	
	{\footnotesize\itshape Министерство науки и высшего образования
		Российской Федерации Федеральное государственное бюджетное
		образовательное учреждение высшего образования «Московский
		государственный технический университет имени Н.~Э.~Баумана
		(национальный исследовательский университет)» (МГТУ им. Н.~Э.~Баумана)
		\\}
	
	\vspace{60mm}
	
	\textbf{ОТЧЕТ}\\
	По лабораторной работе №7\\
	По курсу: «Анализ алгоритмов»\\
	Тема: «Алгоритмы поиска подстроки в строке»\\
	
	\vspace{60mm}
	
	\hspace{70mm} Студент:\hfill Козаченко~А.~А.\\
	\hspace{70mm} Группа: \hfill ИУ7-54Б\\
	%\vspace{5mm}
	\hspace{70mm} Преподаватели:\hfill Волкова~Л.~Л.,\\
	\hfill Строганов~Ю.~В.\\
	
	\vfill
	Москва, 2020
\end{titlepage}


\chapter*{Введение}
\addcontentsline{toc}{chapter}{Введение}

Поиск подстроки в строке — одна из простейших задач поиска информации.
Применяется в виде встроенной функции в текстовых редакторах, СУБД, поисковых машинах, языках программирования и т. п. \cite{Gusfield}

\section*{Задачи работы}

Цель лабораторной работы: изучить алгоритмы поиска подстроки в строке.

В рамках выполнения работы необходимо решить следующие задачи:
\begin{itemize}
	\item изучить стандартный алгоритм, алгоритмы Кнута — Морриса — Пратта и Бойера — Мура;
	\item реализовать данные алгоритмы;
	\item привести подробное описание работы каждого алгоритма.
\end{itemize}


\chapter{Аналитическая часть}

\section{Описание задачи}

Пусть даны строки \textit{source} и \textit{pattern}, обозначим их \textit{s} и \textit{p} соответственно.
Необходимо проверить, входит ли строка \textit{p} в \textit{s}, если да, то найти индекс первого вхождения.

\section{Описание алгоритмов}

\subsection{Стандартный алгоритм}

Стандартный алгоритм основан на последовательном сравнении всех подстрок строки \textit{s} с \textit{p}, т. е. будет происходить сравнение всех подстрок размера $|p|$, начиная с индексов $i = 1,2,...,|s|-|p|+1$.

Пусть $s= "abcabccba"$, $p = "cab"$.
В таблице \ref{tbl:standard-analysis} показаны сравнения символов, выполняемые в ходе работы алгоритма.

\begin{table}[h]
	\begin{center}
		\begin{tabular}{|c|c|c|c|c|c|c|c|c|c|}
			\hline
			№&a&b&c&a&b&c&c&b&a\\
			\hline
			1&c&a&b& & & & & &\\
			\hline
			2& &c&a&b& & & & &\\
			\hline
			3& & &c&a&b& & & &\\
			\hline
		\end{tabular}
	\end{center}
	\caption{Сравнение символов в стандартном алгоритме}
	\label{tbl:standard-analysis}
\end{table}

\subsection{Алгоритм Кнута — Морриса — Пратта}

Алгоритм Кнута — Морриса — Пратта является оптимизацией стандартного алгоритма \cite{Smyth}.
Необходимо дать определения префикса, суффикса и префикс-функции.

Алгоритм был разработан Д. Кнутом и В. Праттом и, независимо от них, Д. Моррисом \cite{Kormen}.
Результаты своей работы они опубликовали совместно в 1977 году \cite{Knuth}.

После частичного совпадения начальной части подстроки $patters$ с соответствующими символами строки $source$ мы фактически знаем пройденную часть строки и может «вычислить» некоторые сведения (на основе самой подстроки $pattern$), с помощью которых потом быстро продвинемся по тексту.

Идея КМП-поиска — при каждом несовпадении двух символов текста и образа образ сдвигается на самое длинное совпадение начала с концом префикса (не учитывая тривиальное совпадение самого с собой) \cite{Okulov}.

Рассмотрим пример. Создается массив сдвигов, таблица \ref{tbl:kmp-shift-analysis}.

\begin{table}[!h]
	\begin{center}
		\begin{tabular}{|c|c|c|l|l|l|}
			\hline
			0 & 1 & 2 & 3 & 4 & 5 \\ \hline
			a & b & c & a & b & d \\ \hline
			0 & 0 & 0 & 1 & 2 & 0 \\ \hline
		\end{tabular}
	\end{center}
	\caption{Массив сдвигов}
	\label{tbl:kmp-shift-analysis}
\end{table}

В таблице \ref{tbl:kmp-example-analysis} представлена работа алгоритма.

\begin{table}[!h]
	\begin{center}
		\begin{tabular}{|l|l|l|l|l|l|l|l|l|l|l|l|l|l|l|l|}
			\hline
			cтрока    & a & b & c & a & b & e                         & a & b & c & a                         & b                         & c                         & a                         & b                         & d                         \\ \hline
			подстрока & a & b & c & a & b & \cellcolor[HTML]{FE0000}d &   &   &   &                           &                           &                           &                           &                           &                           \\ \hline
			подстрока &   &   &   & a & b & \cellcolor[HTML]{FE0000}c & a & b & d &                           &                           &                           &                           &                           &                           \\ \hline
			подстрока &   &   &   &   &   & \cellcolor[HTML]{FE0000}a & b & c & a & b                         & d                         &                           &                           &                           &                           \\ \hline
			подстрока &   &   &   &   &   &                           & a & b & c & a                         & b                         & \cellcolor[HTML]{FE0000}d &                           &                           &                           \\ \hline
			подстрока &   &   &   &   &   &                           &   &   &   & \cellcolor[HTML]{34FF34}a & \cellcolor[HTML]{34FF34}b & \cellcolor[HTML]{34FF34}c & \cellcolor[HTML]{34FF34}a & \cellcolor[HTML]{34FF34}b & \cellcolor[HTML]{34FF34}d \\ \hline
		\end{tabular}
		\caption{Пример работы алгоритма КМП}
		\label{tbl:kmp-example-analysis}
	\end{center}
\end{table}

\subsection{Алгоритм Бойера — Мура}

Алгоритм поиска строки Бойера — Мура считается наиболее быстрым среди алгоритмов общего назначения, предназначенных для поиска подстроки в строке.
Был разработан Бойером и Муром в 1977 году \cite{Boyer}.
Преимущество этого алгоритма в том, что ценой некоторого количества предварительных вычислений над шаблоном (но не над строкой, в которой ведётся поиск) шаблон сравнивается с исходным текстом не во всех позициях — часть проверок пропускаются как заведомо не дающие результата.

Основная идея алгоритм — начать поиск не с начала, а с конца подстроки.
Наткнувшись на несовпадение, мы просто смещаем подстроку до самого правого вхождения данного символа, не учитывая последний.

Рассмотрим пример. Создаётся массив прыжков, таблица \ref{tbl:bm-shift-analysis}.

\begin{table}[!h]
	\begin{center}
		\begin{tabular}{|c|c|c|l|l|l|}
			\hline
			a & b & c & a & b & d \\ \hline
			2 & 1 & 3 & 2 & 1 & 6 \\ \hline
		\end{tabular}
	\end{center}
	\caption{Массив прыжков}
	\label{tbl:bm-shift-analysis}
\end{table}

В таблице \ref{tbl:bm-example-analysis} представлена работа алгоритма.

\begin{table}[!h]
	\begin{center}
		\begin{tabular}{|l|l|l|l|l|l|l|l|l|l|l|l|l|l|l|l|}
			\hline
			cтрока    & a & b & c & a & b & e                         & a & b & c & a                         & b                         & c                         & a                         & b                         & d                         \\ \hline
			подстрока & a & b & c & a & b & \cellcolor[HTML]{FE0000}d &   &   &   &                           &                           &                           &                           &                           &                           \\ \hline
			подстрока &   &   &   &   &   &                           & a & b & c & a                         & b                         & \cellcolor[HTML]{FE0000}d &                           &                           &                           \\ \hline
			подстрока &   &   &   &   &   &                           &   &   &   & \cellcolor[HTML]{34FF34}a & \cellcolor[HTML]{34FF34}b & \cellcolor[HTML]{34FF34}c & \cellcolor[HTML]{34FF34}a & \cellcolor[HTML]{34FF34}b & \cellcolor[HTML]{34FF34}d \\ \hline
		\end{tabular}
	\end{center}
	\caption{Пример работы алгоритма БМ}
	\label{tbl:bm-example-analysis}
\end{table}


\section*{Вывод}

В данной работе стоит задача реализации алгоритмов поиска подстроки в строке.


\chapter{Конструкторская часть}

\section{Разработка алгоритмов}

На рисунках \ref{img:kmp} и \ref{img:bm} представлены схемы алгоритмов КМП и БМ соответсвенно.

\imgext{height=170mm}{kmp}{pdf}{Схема алгоритма Кнута — Морриса — Пратта}

\imgext{height=160mm}{bm}{pdf}{Схема алгоритма Бойера — Мура}

\section*{Вывод}

Были разработаны алгоритмов поиска подстроки в строке.


\chapter{Технологическая часть}

В данном разделе приведены средства реализации и листинг кода.

\section{Средства реализации}

В качестве языка программирования для реализации данной лабораторной работы был выбран высокопроизводительный язык C++ \cite{cpp17}, так как он предоставляет широкие возможности для эффективной реализации алгоритмов.

\section{Листинг кода}

В листингах \ref{lst:standard}, \ref{lst:kmp} и \ref{lst:bm} приведены листинги алгоритма сортировки пузырьком, вставками и выбором соответственно.

\begin{lstinputlisting}[
	caption={Стандартный алгоритм},
	label={lst:standard},
	style={cpp},
	linerange={6-26}
	]{../source/string/search.cpp}
\end{lstinputlisting}

\begin{lstinputlisting}[
	caption={Алгоритм Кнута — Морриса — Пратта},
	label={lst:kmp},
	style={cpp},
	linerange={28-68}
	]{../source/string/search.cpp}
\end{lstinputlisting}

\begin{lstinputlisting}[
	caption={Алгоритм Бойера — Мура},
	label={lst:bm},
	style={cpp},
	linerange={70-112}
	]{../source/string/search.cpp}
\end{lstinputlisting}

\section{Тестирование функций}

В таблице~\ref{tbl:test} приведены тесты для функций, реализующих алгоритмы поиска подстроки в строке.
Тесты пройдены успешно.

\begin{table}[h!]
	\begin{center}
		\begin{tabular}{|c|c|c|}
			\hline
			\textbf{Строка1} & \textbf{Строка2} & \textbf{Ожидаемый результат} \\ \hline
			aaaaa      & a  & 1  \\ \hline
			aapo       & po & 2  \\ \hline
			poaa       & po & 0  \\ \hline
			qwerty     & a  & -1 \\ \hline
			popopopopo & po & 0  \\ \hline
		\end{tabular}
		\caption{Тестирование функций}
		\label{tbl:test}
	\end{center}
\end{table}

\section*{Вывод}

Правильный выбор инструментов разработки позволил эффективно реализовать алгоритмы, настроить модульное тестирование и выполнить исследовательский раздел лабораторной работы.


\chapter{Исследовательская часть}

\section{Примеры работы}

На рисунках \ref{img:example1} и \ref{img:example2} изображены примеры работы программы.

\boximg{50mm}{example1}{Пример 1 работы программы}

\boximg{50mm}{example2}{Пример 2 работы программы}

\section{Результаты тестирования}

Проверяем нашу программу на тестах из таблицы \ref{tbl:test}. Полученные результаты представлены в таблице \ref{tbl:test-results}.

\begin{table}[h!]
	\begin{center}
		\begin{tabular}{|c|c|c|c|}
			\hline
			\textbf{Строка1} & \textbf{Строка2} & \textbf{КМП} & \textbf{БМ}  \\ \hline
			aaaaa      & a  & 0  & 0  \\ \hline
			aapo       & po & 2  & 2  \\ \hline
			poaa       & po & 0  & 0  \\ \hline
			qwerty     & a  & -1 & -1 \\ \hline
			popopopopo & po & 0  & 0  \\ \hline
		\end{tabular}
	\end{center}
	\caption{Результаты тестирования программы}
	\label{tbl:test-results}
\end{table}

\section*{Вывод}

В данном разделе были приведены примеры работы и протестирована программа.

\chapter*{Заключение}
\addcontentsline{toc}{chapter}{Заключение}

По итогам аналитического раздела были описаны стандартный алгоритм, алгоритм Кнута — Морриса — Пратта и алгоритм Бойера — Мура для нахождения подстроки в строке.

\addcontentsline{toc}{chapter}{Литература}
\bibliographystyle{utf8gost705u}  % стилевой файл для оформления по ГОСТу
\bibliography{51-biblio}          % имя библиографической базы (bib-файла)


\end{document}
